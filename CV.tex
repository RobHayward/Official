 
\documentclass[margin,11pt]{res} % default is 10 pt
% margin option puts section titles to left of text
%\usepackage{helvetica} % uses helvetica postscript font (download helvetica.sty)
%\usepackage{newcent}   % uses new century schoolbook postscript font 

\begin{document}
\name{ROB HAYWARD\\[12pt]} % the \\[12pt] adds a blank line after name

\address{{\bf 68 Osborne Villas} \\  Brighton \\ East Sussex \\ BN3 2RB\\
        (0)1273 726 154}


\begin{resume}


\section{Employment}  
     {\bf University of Brighton Business School}  (2003 - \\
     Senior Lecturer\\
\begin{itemize}
\item Responsible for teaching under-graduate and post-graduate economics and finance.  
\item Specialisation in the economics of financial markets and international economics.   
\item Course Leader for MSc Finance.  Formerly course leader for  BSc Finance \& Investment and BSc. Economics \&Finance courses.  
\item Module leader for EC381 Financial and Capital Markets; EC387 Applied Financial Techniques; ECM22 Market-making and investment strategies; EC224 Economic Theory and Application.
\item Achievements include leading the group that conducted the review of the MSc Finance suite of programmes, developing new pathways in Finance \& Risk Management, Finance \& Banking as well as Finance \& Accounting; presenting for validation and defending before validation committee; working with marketing team to identify degrees popular with international students.  
\item Being part of the team that reviewed the BSc Finance and Investment degree.  This involved updated module specifications, mapping learning outcomes to QAA benchmarks as well as presenting and defending materials before internal and external reviewers.  
\item Helping to set up the University of Brighton \emph{trading room}.  This included sourcing and selecting software and incorporating the learning space into the curriculum. 
\end{itemize}

%\section{Employment} 
 %    {\bf Laureate Education - University of Liverpool} (2004 - \\
 % Faculty Manager \\
%\begin{itemize}
%\item Charged with managing on-line faculty including reviewing classes, coaching and advising faculty as well as managing the Investment Strategies module for the joint Laureate-University of Liverpool on-line MBA.
%\item   Achievements include managing the review of the Investment Strategies module and being presented with an award for Outstanding Achievement in 2012 for work on coaching faculty.  
%\end{itemize}

\section{Education}
 {\bf PhD} (2007 - 2013)\\
   University of Brighton\\
\begin{itemize}
\item Analysis of the relationship between speculation and risk in the foreign exchange market.  There are three strands to the research:  speculation is measured by using options and regulatory positions in the futures market; an analysis of the carry-trade is used to provide a risk-based solution to the uncovered-interest-parity puzzle; a structural vector auto-regression (SVAR) model of capital flows, including speculation, and the real exchange rate is developed.  
\end{itemize}
%Speculation is found to have a positive correlation with the risk of a rapid reversal (or 'crash risk').  The main contributions are in the measurement or speculation, the analysis of the carry trade and the augmentation of portfolio flows model with speculative flows.  

\section{Education}
 {\bf MSc Economics} (1988-1990)\\
  University College London\\
\begin{itemize}
\item Achieved a good pass and completed modules in International Economic, Monetary Economics and Econometrics.  
\end{itemize}

\section{Publications}
	UIP, the carry trade and Minsky's Financial Instability Hypothesis in the CEE and CIS in {\bf Poland and the Eurozone}, Studies in Economic Transition, Palgrave

\section{Conferences}
 	{\bf EACES Annual Conference} (4th - 6th September 2014) \\
Presented paper \emph{Markov-switching model of financial stability}\\
	{\bf Poland and the Eurozone} (19th-20th September 2013) \\
Presented paper \emph{Crash risk and the carry trade: an analysis of UIP in CEE and CIS}\\
    {\bf EACES Annual Conference} (6th - 8th September 2012) \\
Presented paper \emph{SVAR model of international capital flow and the real exchange rate}\\
    {\bf EACES Conference} (September 2011)
Presented paper \emph{The Carry Trade is CEE and CIS}
 
%\section{Employment} 
  %   {\bf Introduction to Trading - ESCP-EAP and Judge Business School Cambridge} (2012 - \\
 % Facilitating classes and managing module \\
% A ran a module called An Introduction to Trading and ESCP-EAP in London which was available for Masters of Finance and Management students.  This involved planning the course, providing lectures and seminar classes using the Rothman Interactive Trader software.  I also facilitated sessions at the Judge Business School on the MBA summer elective programme. 
 
\section{Other relevant skills}
\begin{itemize}
 \item Good working knowledge of open-source \textbf{software package R}
 \item Incorporated \textbf{RIT trading simulator} into the Finance module
\item Good working knowledge of open-source typesetting package \textbf{LaTex}
\item Use of \textbf{Eviews} in teaching entry level econometrics
\item Basic knowledge of \textbf{Octave}, the open source version of Matlab
\end{itemize}

\section{Employment}      
     {\bf RedTower Research} (2003-2007  \\
                Economic Consultant \\ 
\begin{itemize} 
\item Responsible for producing weekly economic report with analysis of upcoming events that would affect the performance of financial markets.  
\end{itemize}
%I was required to present monthly strategy reports that would identify investment opportunities in foreign exchange, bond and money markets.  

\section{Employment}
 {\bf ABN Amro} (2001 - 2003)\\
Senior Foreign Exchange Strategist\\
\begin{itemize} 
\item  Charged with making and presenting forecasts for exchange rates and creating strategies for bank customers.   
\item Presentations were made to customers including Moore Capital, Allianz Insurance, Dutch and Swedish pension funds. 
\item  I was also responsible for maintain good contact with the media, raising the profile of ABN Amro currency research by conducting television and radio interviews.  Interviews included Channel Four news, the Today Programme, Bloomberg and CNBC.   
\item Responsible for creating and updating statistical models for exchange rate forecasting.  
\end{itemize}

\section{Employment} 
  {\bf Bank of America} (1994 - 2000) \\
Senior Economic Advisor \\
\begin{itemize}
\item Responsible making and presenting economic and financial forecasts for the major economies to bank customers and the media. 
\item Bank of America weekly report was regularly praised as being in the within the highest category of its class by customers and media organisations. Contributed to the increase in the exposure of Bank of America capital markets group with over 100 tv and radio interviews in 1999.     
\item Turning economic and financial market forecasts into trading, hedging and allocation ideas for the bank and the bank’s customers. 
\item Working with foreign exchange and interest rate sales and trading teams to turn ideas and forecasts into profitable strategies. 
\end{itemize}

\section{Education}  
  {\bf Post Graduate Certificate in Teaching and Learning} (2004)\\
   University of Brighton\\


\section{Education}
 {\bf BA (Hons) Politics and Government} (1979-1982)\\
  City of London Polytechnic\\
Achieved a 2:1 in the study of the philosophy, history and economics of political systems.    

%\section{Education} 
 %{\bf A Levels} (1979-80)\\
 % Sudbury Upper School\\
 % A-Levels in English, Maths and History.  

%\section{Education} 
 %{\bf O Levels} (1979-80)\\
 % Great Conrnard Upper School\\
 % 12 O Levels.   
                 

\end{resume} 
\end{document}





